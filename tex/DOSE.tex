\outline{1}{Diminishing Returns of Dose and Duration of Rehabilitation Post-Stroke as Revealed by Dynamic Mixed Effect Model}
\chapter{Diminishing Returns of Dose and Duration of Rehabilitation Post-Stroke as Revealed by Dynamic Mixed Effect Model}
\label{dosechapter}


\outline{2}{Introduction}
\section{Introduction}
In this chapter, I present my work on the data from the DOSE clinic trial.


\section{Method}
\subsection{DOSE Trial}
In this study, we analyzed the data from Dose Optimization for Stroke Evaluation (DOSE) clinical trial \cite{}. 
40 post-stroke individuals participated in this study. 
Fig.1 shows the timeline of the study: participants receive two baseline tests before three periods of therapy intervention, followed by 6 follow-up tests. 
Each period of therapy intervention lasts 1 week, right after a pre-training test and followed by a post-training test. 
There is one month between intervention periods and follow-up tests. 
The whole study lasts 37 weeks for each participant.
  
Participants were grouped randomly into four groups with different dosage of therapy training during the intervention period: zero, low, moderate, and high therapy dose, measured by the amount of time of treatment: 0, 5, 10, 20 hours per week.

In each test, participants received Wolf Motor Function Test (WMFT) and Bilateral Arm Reaching Test (BART), as well as an interview of Motor Activity Log (MAL). 
Concerning the BART and WMFT tests last as long as 2 hours, we include this testing effect into the hours of training, so that we have 22, 12, 7, 2 hours of training for the four groups.

\section{State Space Model to Nonlinear Mixed Effect Model}
In the field of motor perturbation adaptation, state space model is adopted to model learning behavior [cite]. 
In particular, the state of the system $ x_k $ at trial $ k $ follows 
\begin{equation}\label{ssm}
	x_{k+1} = Ax_k + Bu_k
\end{equation}
Where $ A $ represents retention, and often takes a value slightly smaller than 1; 
where as B, slightly bigger than 0, represents learning due to input $ u_k $ {often indicating the error in motor perturbation adaptation).
This formula can be written as a differential equation:
\begin{equation}
	\frac{\Delta x_k}{\Delta t} = \frac{x_{k+1}-x_k}{\Delta t} = \frac{(A-1)x_k}{\Delta t} + \frac{Bu_k}{\Delta t}
\end{equation}
Omit index $ k $ and let $ a = (A-1)/\Delta t $, $ b = B/\Delta t $ and $ \Delta t\rightarrow 0 $, we have
\begin{equation}\label{oode}
	\frac{dx(t)}{dt} = ax(t)+bu(t)
\end{equation}
The solution to this differential equation is in general
\begin{equation}\label{generalsolution}
	x = x_0e^{at} + b\int_0^t e^{a(t-\tau)}u(\tau)d\tau
\end{equation}
If $ a $ is small comparing to the time scale concerned (that is to say $ |at_{\text{max}}| \ll 1 $) and $ b $ is comparibly small to $ a $ (that is to say $ \text{lim}\frac{bu_\text{max}}{ax_\text{max}} \leqslant \text{const.} $), then the first order approximation to this solution is
\begin{equation}\label{specialsolution}
	x = x_0 + x_0at + b \int_0^t u(\tau)d\tau
\end{equation}
Let $ c=x_0 a $, the original differential equation \ref{oode} becomes
\begin{equation}
	\frac{dx(t)}{dt} = bu(t) + c
\end{equation}
To isolate the effect of training input $ u(t) $, we further assume that c can be ignored when $ u(t) \neq 0 $, i.e.
\begin{equation}\label{fode}
	\frac{dx(t)}{dt} = bu(t) + cv(t)
\end{equation}
where
\begin{equation}
v(t) = 
\begin{cases}
	1, & \text{if}\ u(t) = 0 \\
	0, & \text{otherwise}
\end{cases}
\end{equation}
Or, in logical notation, $ v(t) = \sim u(t) $. 
This assumption would be further justified later when we show $ bu(t) $ is much larger than $ c $.

Denote the solution to Eqn. \ref{fode} as nonlinear function $ x=f(x_0,b,c,t) $, we estimate the mixed effect model as
\begin{equation}\label{eqn:mixedeffect}
	x_i (t)=f(x_0i,b_i,c_i,t) + \epsilon_i (t)
\end{equation}
where $ x_0i $, $ b_i $, $ c_i $ are the mixed effect parameters which can vary from individual to individual (cite Bates).
$ \epsilon_i (t) $ is the residual. 
We choose different mixed effects on $ b_i $ and $ c_i $ to investigate different aspects of training mentioned above.

\textbf{Training Efficacy and Efficiency.}
To estimate the efficacy of training (the overall effect of training), we let $ u(t)=1 $ during the intervention periods as shown in Fig.2A. 
To estimate the efficiency of training (the effect of training per hour), we let $ u(t) $ to be the number of treatment hours corresponding to the group, as shown in Fig.2B. 
To investigate how training efficacy and efficiency depends on dosage of training, we use a linear formula for the mixed effects of $ b_i $ and $ c_i $:
\begin{eqnarray}
	b_i &=& \beta_1^b + \beta_2^b D_i + \delta_i^b   \\
	c_i &=& \beta_1^c + \beta_2^c D_i + \delta_i^c   \\
	x_{0i} &=& \beta_1^{x_0} + \beta_2^{x_0} F_i + \delta_i^{x_0}
\end{eqnarray}
where $ D_i $ is the dosage of training (number of hours per week) participant $ i $ received, 
$ F_i $ is the initial FM score of participant $ i $, 
$ \beta $ are fixed effects and $ \delta_i $ are random effects. 
Indexes on the shoulder indicates the parameter to which this fixed or random effect belongs.

In the case when we take the dosage of training as categorical variable, i.e. how efficacy and efficiency depends on the four groups, the formula becomes
\begin{equation}
	b_i = \beta_1^b + \beta_2^b D_{i2} + \beta_3^b D_{i3} + \beta_4^b D_{i4} + \delta_i^b
\end{equation}
Where $ D_{ij} $ is 1 when participant i belongs to group j, 0 otherwise. The formula for $ c_i $ is in the same structure.

\textbf{The Timing of Training}
To look at the effect of training at different period, we further assume $ b $ and $ c $ can take different values at different intervention periods. 
We applied this assumption in two ways: 1) A linear dependence of $ b $ (and $ c $) on training period $ k (k=1,2,3) $, namely
\begin{eqnarray}
	b(k) &=& b_0 + b_t (k-1) \\
	c(k) &=& c_0 + c_t (k-1)
\end{eqnarray}
So that we have two more parameters than Eqn. \ref{fode}. 

Due to insufficient data points (too few data points at each training period), we simplify the mixed effects of $ b(k) $ and $ c(k) $:
\begin{eqnarray}
	b_0 &=& \beta_1^b + \delta_i^b \\
	b_t &=& \beta_2^b
\end{eqnarray}
The formula for $ c_0 $ and $ c_t $ is in the same structure.

2) Independent values of $ b $ and $ c $ at different training period, namely
\begin{equation}
	b = b_k, c = c_k, k = 1,2,3
\end{equation}
So that we have four more parameters than Eqn. \ref{fode}. 

We use maximum likelihood to estimate parameters.
