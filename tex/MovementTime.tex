\chapter{The Duration of Reaching Movement is Longer than Predicted by Minimum Variance}
\label{cha:md}

\section{Introduction}

\section{Methods - from the paper, need to add}
Eleven young adult right-handed volunteers (between 20 and 30 years old; 3 women, 8 men) with no declared neurological impairments gave written informed consent to participate in the experiment described below, approved by the University of Southern California Ethics Committee.

Subjects performed reaching movements with their right arms to one of two targets at 45° and 135° (relative to the rightward direction parallel to the body) without online visual feedback in four conditions: preferred, fast, medium, and slow (Fig. 1, A and C).  
Hand movements were recorded via a digital pen moving on a tablet (Wacom Tech). 
The tablet height was positioned such that the arm was moving approximately in a two-dimensional horizontal plane, with movements primarily involving horizontal shoulder and elbow rotations. 
The home position was placed such that the shoulder formed approximately a 45° angle with the coronal plane and the elbow angle at 90°. 
The distance from home position to targets was 7.4 cm. 
Target directions (45° and 135°) were selected to maximize the difference in inertia at the hand (see Gordon et al. 1994). 
To minimize fatigue due to the large number of trials, the forearm was supported with a sling attached to a \~4-meter-long cable hanging from the ceiling. 
Because of the long cable length, the pendulum effect was small.

At the begining of each trial, subjects saw the home position, the target, and the cursor presented on a computer screen facing the subjects. 
Subjects were instructed to keep the cursor within the home position, and to move toward the target after a “go” signal at which moment all visual feedback was cleared. 
Final cursor position was displayed for 1 second after the recording period (900 ms for fast, 1,400 ms for intermediate, 2,400 ms for preferred and slow movement conditions). 
In the preferred condition, subjects reached for the targets as accurately as possible using their own selected movement duration (they were instructed to “reach naturally”). 
In the fast, medium, and slow conditions, subjects reached for the targets quickly ($\sim$350 ms), at a medium speed (400--800 ms), and slowly ($\sim$1,200 ms), respectively (Fig. 1B). 
When the movement duration was outside of the range, a “too fast” or “too slow” message was displayed and the trial was repeated. 

Subjects first practiced all conditions in a familiarization session with online visual feedback of the cursor position, with 25 trials in each condition. 
To avoid possible carryover effects, the preferred movement duration condition was performed first, with the order of the two targets counterbalanced across subjects (Fig. 1C). 
Subjects performed 120 trials for each target in the preferred condition and then 80 trials in the other six conditions (3 movement durations, fast, medium, slow for both targets), which were counterbalanced across subjects. 
The numbers of trials were determined in a pilot study, in which we noted that movement duration and final error stabilized only after a few dozen trials and that more trials were needed in the preferred condition for this stabilization to occur. 

[Move this to Stats methods] This was confirmed in preliminary mixed-model analysis of the actual experiment, which showed a significant effect of a trial covariate on movement duration when all trials were considered. This effect disappeared (P > 0.05) when only the last 50 trials in each condition and target were analyzed. Thus, for each subject, 400 movements were analyzed (50-trial block for each target in the 4 duration conditions).