\chapter{The Duration of Reaching Movement of Healthy Individuals}
\label{cha:ocmd}

\section{Introduction}
After more than a century of research, we gained a significant amount of knowledge on the planning and kinematics of human reaching movements. [read Kandel] For the planning of reaching, [Gordon] and others .... Whereas theories and models such as [minimum jerk, torque, acceleration], optimal control, accounts for the kinematics of reaching. However, despite these works, we are still unclear on how we select the duration of arm reaching movements. 

Fitts formulize the relationship between movement duration and task difficulty  “Fitts’ law” (Fitts, 1954) empirically from experiment data, which is a well-known empirical law addressing the speed-accuracy tradeoff. (Harris and worpert 1998, 2006) showed that with the presence of Signal Dependent Noise (SDN), the Fitts law can be derived from minimum variance principle. However, although their model successfully accounts for trajectories and durations of saccade and arm reaching movements, it fails to explain why subjects, given a specific task, don't move slower than they do to achieve smaller endpoint variability. Moreover, the model cannot distinguish minimizing endpoint variability due to SDN, from minimizing effort, because most of the time(footnote) their formulation leads to equivalant mathmetical expressions, as shown in [..]. 

If a task does not specifically (explicitely) have a varibility limit, then minimizing endpoint variance with SDN or minimizing effort inevitably leads to infinitely long movement duration. Therefore, additional mechanism is needed to reduce the movement duration. [Shadmehr 2010, 2016] proposes that the movement duration pose a discount of reward of the goal, which is called the "Cost of Time". Although Although this hypothesis has its base on physiological studies (Johnson  Bickel, 2002), but the discounting time constant is in the range of days or months. Whether discounting works within seconds remains to be confirmed. Another mechanism to reduce movement duration is to consider Constant Noise (CN). In Van Beers work, it has been shown that the effect of CN is not negligible ... and [authors] shows that CN in optimal control model, leads to larger variability at longer duration. Thus minimizing endpoint variability due to both signal-dependent and constant noise would yield a minimum that could determine movement duration. Correspondingly, our first hypothesis is that the CNS chooses intermediate reaching movement durations to minimize endpoint variability in the face of signal-dependent and constant noise.

[paragraph about the second hypothesis] [from the published paper, need to add more/change] While minimizing end-point variability due to both signal- dependent and constant motor noise would be sufficient to determine the duration of movements, the results of Burdet et al. (2001) suggest that effort minimization contributes to movement planning in addition to noise. In that experiment, training arm reaching movements in a divergent force field led to reduced end-point variability by increasing cocontraction of antagonist muscles. If the CNS only minimized end-point variability, then it would use this energetically costly cocon- traction strategy to reduce end-point error (Osu et al. 2004). The fact that habitual movements (performed without force field) yield more relaxed muscle activity than in the experiment of Burdet et al. suggests that the CNS also minimizes effort. Minimization of effort independently from minimization of end-point variability due to signal-dependent and constant noise sources would lengthen movement duration.  -- -- Our second hypothesis is therefore that the CNS chooses intermediate durations that are longer than those minimizing variability. To test these two hypotheses and examine how humans select movement durations, we conducted an arm movement experiment without visual feedback and measured end-point variability in movements of four duration conditions, fast, medium, slow, and preferred, in which subjects choose to move at their preferred speed. The results verify our hypotheses, providing evidence for the concurrent minimization of effort and variability due to signal-dependent and constant motor noise.

\section{Open-loop Stochastic Optimal Control Model}
To test our hypothesis, we built an open-loop stochastic optimal control model with both SDN and CN, to investigate the relationship between movement duration and endpoint variability (and effort?).

The reaching movement is simulated with a planar arm with a shoulder and an elbow, whose joint angles are represented by $\bm{q} = (q_1, q_2)^T$, controlled by the open-loop stochastic optimal controller. 
The forward and inverse kinematics are formulated as in (Van Beers), and the dynamics equation is

\begin{equation} \label{maindynamics}
	\ddot{\bm{q}} = \bm{M}(\bm{q})^{-1} (\bm{\tau} - \bm{C}(\bm{q}, \dot{\bm{q}}) - \bm{B}\dot{\bm{q}})
\end{equation}

where $\bm{M}(\bm{q})$ is the inertia matrix, $\bm{C}(\bm{q}, \dot{\bm{q}})$ accounts for the centrifugal force and the Coriolis effect, and $\bm{B}$ is the viscosity matrix, which we assume to be a diagonal matrix with a single value $b$ on the diagonal. $b$ is a free parameter as discussed above. 
See Appendix \ref{app:oc} for matrices definitions and parameter values.

We use a second-order linear filter as the muscle model (as in Van Beers) which receives motor commands and gives torques over joints:

\begin{equation}
	\bm{u} = t_et_a\ddot{\bm{\tau}} + (t_e+t_a)\dot{\bm{\tau}} +\bm{\tau}
\end{equation}

where $\bm{u}$ is the motor command, $t_e$, $t_a$ are time constants for excitation and activation respectively. 

The motor command $\bm{u}$ is corrupted with signal-dependent noise and constant noise:

\begin{equation}\label{cnsdn}
u_t \rightarrow (1 + \epsilon\sigma_{\text{SDN}}) u_t + \xi\sigma_{\text{CN}}
\end{equation}

where $\epsilon$ and $\xi$ are random variables from the normal Gaussian distribution.
$\sigma_{\text{CN}}$ and $\sigma_{\text{SDN}}$ represent the levels of noise.

To circumvent the difficulty of solving a nonlinear optimal control problem, we discritize and linearize the kinematics and dynamics equation with Newton's method [Newton] with 10 ms time step interval around an approximated solution (as in Li 2004, Appendix \ref{app:oc}), giving rise to a linear-quadratic problem with the following cost to minimize:

\begin{equation}
V = \bm{s}_T^T\bm{Q}_T\bm{s}_T + \sum_{t=1}^T\bm{u}_t^T\bm{Ru}_t
\end{equation}

where $\bm{s_t}$ is the state vector. 
$\bm{Q}$ is arranged in such a way that the first term represents the square of endpoint error in hand space (Appendix \ref{app:oc}). 
$\bm{R}$ is a diagonal matrix with a single value $w_r$ on the diagonal.
$w_r$ is a free parameter.

[Leave the discussion about linearize and Gaussian distribution propogation to appendix]


\section{Estimation of SDN and CN in Reaching Movements}
Although $\sigma_{\text{SDN}}$ and $\sigma_{\text{CN}}$ (see equation \ref{cnsdn}) have previously been estimated {van Beers, 2004 }, these estimates were based on inverse dynamics control of a simulated arm with a viscosity parameter that was estimated at rest from restoring forces to small perturbations {Gomi, 1998 3}. 
However, the amount of viscosity during movements is presumably substantially smaller during movement than at rest {Burdet, 2013 4}. 
In addition, we found in simulations (Appendix \ref{app:oc}) that viscosity has a profound effect on the distribution of endpoints.  
We therefore re-estimated $\sigma_{\text{SDN}}$ and $\sigma_{\text{CN}}$ with viscosity as a free parameter by fitting an optimal open-loop controller to reaching data described in Chapter \ref{cha:md}.

As shown in equation \ref{cnsdn}, CN and SDN is assumed to be independent and Gaussian. 
The endpoint distribution due to motor noise is generally not Gaussian because of nonlinearity of the kinematics and dynamics.
However, since we linearized the system, endpoint distribution then becomes Gaussian, and therefore can be summarized through the first and second moments of the distribution. 
Please see Appendix \ref{app:oc} for a comparison between endpoint distributions from the original dynamics and the linearized dynamics.

We picked the $\sigma_{\text{SDN}}$ and $\sigma_{\text{CN}}$ values that minimize the following fitting error function:

\begin{equation}
\sum_{t}\sum_{c} (\text{TE}_{\text{data},t,c} - \text{TE}_{\text{sim},t,c})^2
\end{equation}

where $t$ and $c$ are indexes for target and condition, and TE stands for Total Error defined as 

\begin{equation}
\text{TE}_{\text{data}} = \frac1N\sum_n(x - x^*)^2 + (y - y^*)^2
\end{equation}

where $n$ is index for trials;

\begin{equation}
\text{TE}_{\text{sim}} = (\bar{x} - x^*)^2 + (\bar{y} - y^*)^2 + \text{tr}(\bm{P})
\end{equation}

where $(\bar{x},\bar{y})$ and $\bm{P}$ are the first and second moment of the simulated endpoint distribution. 
The expressions of TE for data and simulation are different because $\text{TE}_{\text{data}}$ is calculated from actual endpoint data, whereas $\text{TE}_{\text{sim}}$ is calculated from the first and second moment of endpoint distribution.
		
We found that for simulated movement that is longer than 1 second, the endpoint variability is very sensitive to CN.
We assume $\sigma_{\text{CN}}$ depends on movement durations, but $\sigma_{\text{SDN}}$ is constant. [need rational and citation] 
The minimization of the fitting error function yielded the following noise and viscosity parameters: $\sigma_{\text{SDN}} = 0.298$ (unit-less), $\sigma_{\text{CN}}$(fast, medium, preferred, slow) = (0.010, 0.037, 0.035, 0.031)(\si{N.m}), and $b=0.110$ (\si{kg.m^2/s}). 
Note that our estimation of viscosity is significantly lower than Van Beers. 

We interpolate $\sigma_{\text{CN}}$ as a function of movement duration using piecewise cubic Hermite interpolation [ref], shown in figure \ref{}.

\section{Simulation of Cost vs. Movement Duration}

For both the \ang{45} and the \ang{135} targets, the stochastic optimal control model with parameters estimated above produced a U-shape relationship between total error and movement duration (Figure \ref{}). The total error reached a minimum at an intermediate movement duration (approximately 600 ms for both targets), and increased either for faster or slower movements. 

The total cost (equation \ref{}) also exhibited a U-shape characteristic. 
For the left target, the total cost curve reached the minimum at (about) 1000 ms, very close to the average movement duration (1053ms)  that the subjects chose at the preferred condition. 
However, for the right target, the total cost curve reached the minimum only a little longer than the movement duration that minimizes the total error, not enough to the average movement duration ( ms?) that the subjects chose at the preferred condition.
The difference stems from the larger effort required for a movement to the target at \ang{135} (which involves both shoulder and elbow rotation) than the target at \ang{45} (which involves roughly to an elbow rotation).





